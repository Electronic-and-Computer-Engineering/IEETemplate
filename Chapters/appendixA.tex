	
	\chapter{Miscellaneous}
	\section{Some Math}
	%\renewcommand{\thepage}{}
	Again, we present you a small mathematical example. Get familiar with this syntax in order to create your own formulas\footnote{\url{https://en.wikibooks.org/wiki/LaTeX/Mathematics}}.
	\begin{equation}
	\varpi_q(k,l)=\begin{cases}
	&1\quad \quad G_q(k,l)>\rho_{\text{th}}\\
	&0\quad \quad \text{otherwise}\\
	\end{cases}
	\end{equation}
	with $\rho_{\text{th}}$ defined as a constant threshold. The remixing error spreading is performed as follows:
	\begin{eqnarray}
	\hat{X}^{(i+1)}_q(k,l)&=&{\varpi}_q(k,l)\left(\tilde{{X}}^{(i)}_q(k,l)+\frac{E^{(i)}(k,l)}{\sum_{q=1}^Q{{\varpi}_q(k,l)}}\right)\nonumber\\
	\tilde{\mathbf{X}}^{(i)}&=&\mathcal{G}(\hat{X}_q^{(i)}),
	\end{eqnarray}
	where $\sum_{q=1}^Q{{\varpi}_q(k,l)}$ accounts for the overal contributions of sources in time frequency error distribution and the remixing error $E^{(i)}(k,l)$ is defined as 
	\begin{equation}
	E^{(i)}(k,l)=Y(k,l)-\sum_{q=1}^{Q}{\hat{X}_q(k,l)}.
	\end{equation}
	
	\begin{mdframed}
		\begin{lstlisting}[language = TeX, caption={The math example}, linewidth=150mm]
		\begin{equation}
			\varpi_q(k,l)=\begin{cases}
			&1\quad \quad G_q(k,l)>\rho_{\text{th}}\\
			&0\quad \quad \text{otherwise}\\
			\end{cases}
		\end{equation}
		with $\rho_{\text{th}}$ defined as a constant threshold. The remixing error spreading is performed as follows:
		\begin{eqnarray}
			\hat{X}^{(i+1)}_q(k,l)&=&{\varpi}_q(k,l)\left(\tilde{{X}}^{(i)}_q(k,l)+\frac{E^{(i)}(k,l)}{\sum_{q=1}^Q{{\varpi}_q(k,l)}}\right)\nonumber\\
			\tilde{\mathbf{X}}^{(i)}&=&\mathcal{G}(\hat{X}_q^{(i)}),
		\end{eqnarray}
		where $\sum_{q=1}^Q{{\varpi}_q(k,l)}$ accounts for the overal contributions of sources in time frequency error distribution and the remixing error $E^{(i)}(k,l)$ is defined as 
		\begin{equation}
			E^{(i)}(k,l)=Y(k,l)-\sum_{q=1}^{Q}{\hat{X}_q(k,l)}.
		\end{equation}
		\end{lstlisting}
		By adding a label to your equation, you will be able to refer to the equation within the text!
	\end{mdframed}

\section{Program code / listing}
Three types of source codes are supported: code snippets, code segments, and
listings of stand alone files. Snippets are placed inside paragraphs and the others as
separate paragraphs the difference is the same as between text style and display
style formulas\footnote{\url{https://en.wikibooks.org/wiki/LaTeX/Source_Code_Listings}}. In the following, we will give you a short introduction on all three code listing types.

\subsection{individual added program code}
\begin{lstlisting}[language = C,caption = {Single code}]
#include <stdio.h>
#define N 10

int main()
{
int i;

puts("Hello world!");

for (i = 0; i < N; i++)
{
	puts("LaTeX is also great for programmers!");
}

return 0;
}
\end{lstlisting}

\begin{mdframed}
	\begin{lstlisting}[language = Tex, caption = {Example of \emph{Single code}},linewidth= 150mm]
	\begin{lstlisting}[caption = {Single code}]
	#include <stdio.h>
	#define N 10
	
	int main()
	{
	int i;
	
	puts("Hello world!");
	
	for (i = 0; i < N; i++)
	{
	puts("LaTeX is also great for programmers!");
	}
	
	return 0;
	}	
	\end {lstlisting}
	\end{lstlisting}
\end{mdframed}

\subsection{Add specific code file}

\lstinputlisting[language=C, caption = {Same code but now we added the code file instead of copying the code into latex}]{code/test.c}

\begin{mdframed}
	\begin{lstlisting}[language = Tex, caption = {Example of \emph{Adding specific code file}}]
		
		\lstinputlisting[language=C, caption = {Same code but now we added the code file instead of copying the code into latex}]{code/test.c}
		
	\end{lstlisting}
\end{mdframed}

\subsection{Scope on specific code file}

\lstinputlisting[language=C, firstline=6, lastline=13, caption = {Scope on specific code file}]{code/test.c}

\begin{mdframed}
	\begin{lstlisting}[language = Tex, caption = {Example of \emph{Scope on specific code file}}]
	
	\lstinputlisting[language=C, firstline=6, lastline=13, caption = {Specific scope on code file}]{code/test.c}
	
	\end{lstlisting}
\end{mdframed}