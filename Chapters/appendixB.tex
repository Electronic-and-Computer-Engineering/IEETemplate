\chapter{Citations}
For any academic/research writing, incorporating references into a document is an important task. Fortunately, LaTeX has a variety of features that make dealing with references much simpler, including built-in support for citing references. However, a much more powerful and flexible solution is achieved thanks to an auxiliary tool called BibTeX (which comes bundled as standard with LaTeX). Recently, BibTeX has been succeeded by BibLaTeX, a tool configurable within LaTeX syntax.

\section{Create your own bibliography}
Creating your own bibliography will take you some time, but if you structure and maintane it properly you will have a powerful tool for your upcoming theses. One way to structure your bibliography is to use \emph{JabRef}\footnote{\url{http://www.jabref.org/}}, where you can roster all resources manually or directly import existing \emph{.bib}- or \emph{BibTex}-Files.

\section{How should a bibliography look like?}
In this file, and during preparation of your thesis, we suggest you to use the IEEE citation standart in your bibliography. Have a look at the following listing as well at the resulting bibliography!

\begin{itemize}
	\item An example for citing a book is \cite{Mendel2007}\cite{MowlaeePejman2016}
	\item Example for citing a journal paper is \cite{MayerEtAl2017}
	\item A Technical Standard is examplified in \cite{Prechtl2006}
	\item Technical Report \cite{Mathworks2017}
	\item lecture notes \cite{Okorn2017}
	\item An example for citing online resources \cite{Goeschka}	
\end{itemize}